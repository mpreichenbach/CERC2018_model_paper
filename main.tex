\documentclass[12pt]{article}
\usepackage{cite}


\begin{document}

\begin{abstract}
	Silver carp (\emph{Hypophthalmychthys molitrix}) are one of four primary species of invasive cyprinids in the Mississippi River Basin. They threaten the ecosystems in which they have been introduced by competing with native planktivores and the early life stages of other native fishes. Additionally, silver carp can injure recreational boaters, thus causing negative economic impacts. To mitigate their spread, non-physical barriers such as electrical barriers, bubble curtains, carbon dioxide, and acoustic deterrents have been installed at locks and dams throughout the Mississippi River Basin. In this article, we focus on acoustic deterrents, and investigate their effect on behavioral changes on silver carp, using data from a 2018 pond study performed at the Columbia Environmental Research Center. We use a modeling approach combining continuous-time correlated random walks with a hidden Markov model for two behavioral states, and include interpolated sound-intensity values from the acoustic deterrent as a covariate. We also investigate the effect of pond-surface temperature and diel effects on silver carp behavior.
\end{abstract}

\title{Modeling Behavioral Changes of Invasive Carp in Response to Acoustic Deterrents}
\author{
	Reichenbach, Matthew \\
	\texttt{matthew.p.reichenbach@usace.army.mil}
	\and
	Sava, Elena \\
	\texttt{elena.sava@usace.army.mil}
	\and
	Maloney, Megan \\
	\texttt{megan.c.maloney@usace.army.mil}
	\and
	Woodley, Christa \\
	\texttt{christa.m.woodley@usace.army.mil}
	\and
	Urbanczyk, Aaron \\
	\texttt{aaron.c.urbanczyk@usace.army.mil}
}
\date{\today}

\maketitle

\section{Introduction}

\section{Modeling Framework}

\section{2018 Columbia Environmental Research Center Pond Study}

\section{Results}

\section{Conclusion}

\bibliographystyle{abbrv}
\bibliography{references}

\end{document}