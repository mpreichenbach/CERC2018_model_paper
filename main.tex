\documentclass[12pt]{article}
\usepackage{cite}


\begin{document}

\begin{abstract}
	Silver carp (\emph{Hypophthalmychthys molitrix}) are one of four primary species of invasive cyprinids in the Mississippi River Basin. They threaten the ecosystems in which they have been introduced by competing with native planktivores and the early life stages of other native fishes. Additionally, silver carp can injure recreational boaters, thus causing negative economic impacts. To mitigate their spread, non-physical barriers such as electrical barriers, bubble curtains, carbon dioxide, and acoustic deterrents have been installed at locks and dams throughout the Mississippi River Basin. In this article, we focus on acoustic deterrents, and investigate their effect on behavioral changes on silver carp, using data from a 2018 pond study performed at the Columbia Environmental Research Center. We use a modeling approach combining continuous-time correlated random walks with a hidden Markov model for two behavioral states, and include interpolated sound-intensity values from the acoustic deterrent as a covariate. We also investigate the effect of pond-surface temperature and diel effects on silver carp behavior.
\end{abstract}

\title{Modeling Behavioral Changes of Invasive Carp in Response to Acoustic Deterrents}
\author{
	Reichenbach, Matthew \\
	\texttt{matthew.p.reichenbach@usace.army.mil}
	\and
	Sava, Elena \\
	\texttt{Elena.Sava@usace.army.mil}
	\and
	Maloney, Megan \\
	\texttt{Megan.C.Maloney@usace.army.mil}
	\and
	Woodley, Christa \\
	\texttt{Christa.M.Woodley@usace.army.mil}
	\and
	Urbanczyk, Aaron \\
	\texttt{Aaron.C.Urbanczyk@usace.army.mil}
}
\date{\today}

\maketitle

\section{Introduction}

	\subsection{Invasive Carp in the Mississippi River Basin}

	Silver carp (\emph{Hypophthalmychthys molitrix}) were introduced into North America in the 1960s and 1970s, and have since become invasive in the Mississippi River Basin. Due to their large population increase \cite{DeBoer2018}, these large-bodied planktivores threaten aquatic ecosystems \cite{Kolar2007, Pegg2002, Sullivan2018} by competing for resources with native planktivores, and with the early life stages of other native fishes \cite{Solomon2016, Phelps2017, Fritts2018}.  Additionally, silver carp can cause injure fisherman and recreational boaters, which can have significant economic impacts \cite{Solomon2016, Chick2020}.

	\subsection{Deterrent Systems}

	To mitigate the economic and environmental impacts of the silver carp, researchers and wildlife managers have developed a number of tools to reduce their population, from marketing them for human consumption (Morgan2018) to the development of novel gears techniques (Collins2015, Butler2019, Ridgeway2020, Chapman2020). Given their wide dispersal throughout the Mississippi River Basin (USFWS2017), researchers have also developed tools to manage the dispersal of silver carp; these include non-physical deterrents like electrical barriers (Clarkson2004), bubble curtains (Zielinski2016), carbon dioxide (Donaldson2016, Cupp2021), and acoustic deterrents (Vetter2015, 2017), and combinations of these (Ruebush2012, Dennis2019). To increase their effectiveness at deterring dispersal and migration, these deterrents have been installed at key pinch-points along the Mississippi River system  where fish passage controlled (for example, a navigation lock (ACRCC2018)).
	
	The electric barrier installed in the Chicago Area Waterways System (CAWS) has significantly mitigated the upstream movement of silver carp (Parker2016). However, electrical deterrents are not species-specific (Reynolds1996, Dolan2003), and thus can impede the migratory behavior of native species. In addition to these spillover ecological effects, electrical deterrents are unsuited to places which humans frequent for recreation and navigation, due to the danger which they pose.
	
	Acoustic deterrents offer the possibility of selectively targeting silver carp, while posing less risk to humans. A number of studies have shown that a broad-frequency $100$hp outboard motor sound repeatedly deterred silver carp, but were not deterred by pure tones (Vetter2015, 2017, Murchy2017). Additionally, Dennis2019 found that a $40$hp sound, and a proprietery sound, deterred bigheaded carp (which silver carp belong to) as they attempted to pass through a flume. In both of these cases, the carps stopped responding to the deterrent by the end of the study, suggesting habituation or fatigue.
	
	These studies indicate that multi-frequency acoustic signals may deter silver carp, but their effectiveness under a more natural setting remains unknown. These boat-motor sounds also come with energetic demands and subsequent wear on the sound projectors which more compatible signals could lessen. To this end, we study the response of silver carp in a pond study to three acoustic stimuli: a $100$hp outboard motor sound, as in (Vetter2015, 2017, Murchy2017), along with two engineered sounds. In addition to studying the response of carp at the onset of the acoustic stimulus, we also study changes to their response over a 72-hour period. We hypothesized that silver carp would exhibit changes to their behavioral state at the onset of the stimulus, but with less marked response over the course of the 72-hour period.

	\subsection{Movement Modeling}
	
	In the last two decades, methods to collect highly detailed data on animal movements have become easily accessible (McConnell et al. 2010, Tomkiewicz et al. 2010, cited in McClintock 2012). This has led to rapid development in state-space models of animal behavior, and of tools to allow non-specialists to implement the models (Johnson2008, McClintock2012, Michelot2016, Whoriskey2017, McClintockMichelot2018paper).
	
	We used hidden Markov models to study the effect of environmental covariates on carp behavior; we were primarily interested in the effects of sound type ($100$hp boat-motor or the engineered signals) and sound intensity on carp behavior. There has been limited research on this topic, but one recent paper (Faulkner et al, unpublished) found that sound-types were associated with changes to the behavioral states of carp. However, the authors did not incorporate a sound intensity covariate.
	
	We performed the data-processing and modeling in R (R Core Team 2022), primarily using the momentuHMM package (McClintockMichelot2018doc) for location estimation and model fitting. We note that momentuHMM relies heavily on the crawl (Johnson2018) and moveHMM (Michelot2016) packages.

\section{2018 CERC Pond Study}

\section{Modeling Framework}

	In this study, we use a state-space model of carp behavior developed in (Johnson2008, McClintock2012, Michelot2016, Whoriskey2017, McClintockMichelot2018paper), which uses a random-walk model to predict carp locations at regular timesteps, followed by a hidden Markov model to infer behavioral states as a function of environmental covariates. Our novel contribution to these models is the incorporation of spatially interpolated sound intensity values; to do this, we made use of the kriging utilities of the automap package (automap2022).

	\subsection{Continuous-time Correlated Random Walk (CTCRW)}
	
	In our pond study (see the following section), the telemetry data has an irregular timestep; however, hidden Markov models require a constant timestep. We use the 

	\subsection{Hidden Markov Models (HMMs)}
	
	\subsection{Covariates}
	
		\subsubsection{Factor Covariates}
		
		\subsubsection{Numerical Covariates}

\section{Results}

\section{Conclusion}

\bibliographystyle{abbrv}
\bibliography{references}

\end{document}